\documentclass[portrait,a0]{sciposter}

\usepackage{amsmath, amsfonts, amsthm, amssymb}
\usepackage{multicol}
\usepackage[english]{babel}
\usepackage{graphicx}
\usepackage{multirow}
\usepackage{amsfonts}
\usepackage{color}
\usepackage{algorithmicx}
\usepackage{tikz}
\usepackage{pgfplots}


\theoremstyle{definition}\newtheorem{definition}{Definition}
\theoremstyle{plain}\newtheorem{example}{Example}
\theoremstyle{plain} \newtheorem{theorem}{Theorem}

\title{Sparse matrix algebra}
\author{Ivo Kubjas}
\institute{Institute of Computer Science, University of Tartu}
\email{ivokub@ut.ee}  % shows author email address below institute

\leftlogo[1]{/home/ivo/Documents/Kool/UT/masters/poster/logos/500px-Tartu_Ülikool_logo.png}  % defines logo to left of title (with scale factor)

\newcommand{\FFF}{\mathbb{F}}
\newcommand{\AAA}{\mathbb{A}}
\newcommand{\MMM}{\mathbb{M}}
\DeclareMathOperator*{\rank}{rank}

\begin{document}
\conference{Algorithmics course 2014}
\maketitle

\begin{multicols}{2}
    \section{Sparse matrices}

    As the naive matrix multiplication algorithm works in a time $O(n^3)$, where
    $n$ is the size of a $n \times n$-dimensional matrix, then multiplication of
    large matrices is unfeasible even with supercomputers.

    \begin{definition}
        A $l$-sparse matrix is a matrix where nonzero values do not exceed
        $l$-th of all elements of the matrix.
    \end{definition}

    Sparse matrices can be found in several areas of mathematics and computer
    science. For example, one can represent a graph as its adjacency matrix.
    Then taking the $i$-th power of the matrix yields the number of connections
    between vertices in the graph. This metric describes the connectivity of the
    graph. This approach is used for calculating PageRank to study the
    importance of a internet page as more important pages tend to have more
    incoming links.

    Depending on the parameters, different algorithms for constructing the
    product of matrices are proposed. We consider two approaches:
    \begin{itemize}
        \item the case where the number of nonzero elements is linear to the
            side size of the matrix, i.e. $m = O(n)$, where $m$ is the number of
            nonzero elements in the matrix and the matrix has dimension $n
            \times n$;
        \item the case where the number of nonzero elements is linear to the
            size of the matrix, i.e. $m = O(n^2)$ with previously defined
            variables.
    \end{itemize}

    The second problem is well studied for the case if the second matrix has a
    single column, or also that it is a vector. However, a matrix can be
    represented as a collection of vectors and combined after into a matrix.

    The first problem was mostly unknown and thus we have implemented a new
    algorithm for multiplying matrices of this kind.

    \begin{definition}
        A coordinate matrix representation (COO) format is a format where nonzero
        elements are stored together with the row and column location.
    \end{definition}

    \begin{example}
        Let the matrix $M$ be
        \[
            M = \left(\begin{matrix}
            0 & 0 & 1 & 0 \\
            2 & 0 & 3 & 0 \\
            0 & 0 & 0 & 4 \\
            0 & 5 & 0 & 0
            \end{matrix}\right)
        \]
        The COO representation is
        \begin{align*}
            V &= [1, 2, 3, 4, 5], \\
            Y & = [2, 0, 2, 3, 1], \\
            X &= [0, 1, 1, 2, 3].
        \end{align*}
    \end{example}

    \begin{definition}
        A compressed row storage representation (CSR) format is a format where
        the column indices and starting points of the columns are stored as
        arrays together with values as a array.

    \end{definition}

    \begin{example}
        Let the matrix $M$ be
        \[
            M = \left(\begin{matrix}
            0 & 0 & 1 & 0 \\
            2 & 0 & 3 & 0 \\
            0 & 0 & 0 & 4 \\
            0 & 5 & 0 & 0
            \end{matrix}\right)
        \]
        The CSR representation is
        \begin{align*}
            V &= [1, 2, 3, 4, 5], \\
            CI &= [2, 0, 2, 3, 1], \\
            RP &= [0, 1, 3, 4, 4]. \\
        \end{align*}
    \end{example}

    As a novel definition, we propose the following hash-table based
    representation of matrices.

    \begin{definition}
        A hash-table matrix representation (HT) format is a format where the
        values are stored together with the value location in a hash table.
    \end{definition}

    \section{Implementation}

    We have implemented all formats in C programming language. The language was
    chosen because it allows for the most low-level manipulation of memory, thus
    allowing for memory access optimization if the algorithms are implemented on
    parallel hardware. The implementation is available at~\cite{algoproject}.

    \subsection{HT representation}

    We study the implementation of HT format more thoroughly as it is to our
    knowledge novel. The data structure was defined as
    \begin{verbatim}
    typedef struct {
        uint len;
        uint cap;
        uint nrows;
        uint ncols;
        val** values;
        location** locations;
        cell *cells;
    } struct_coo;
    \end{verbatim}

    We implemented the values and locations as list of pointers as it allows for
    in-place modification without expensive retrieval and assignment.

    We used the \textsc{uthash} library for implementing the hash tables.

    The addition and multiplication algorithms were defined as
    
    \subsubsection{Addition algorithm}

    \begin{verbatim}
    HASH_ITER(hh, mat1->cells, item, tmp) {
        coo_set_value(res, *item->value,
                      item->loc.r, item->loc.c);
    }
    HASH_ITER(hh, mat2->cells, item, tmp) {
        v = coo_get_value(res, item->loc.r, item->loc.c);
        if (v) {
            *v += *item->value;
        } else {
            coo_set_value(res, *item->value,
                          item->loc.r, item->loc.c);
        }
    }
    \end{verbatim}

    \subsubsection{Multiplication algorithm}

    \begin{verbatim}
    for (i = 0; i < mat1->len; i++) {
        location* loc1 = mat1->locations[i];        
        val* val1 = mat1->values[i];
        for (j = 0; j < mat2->len; j++) {
            location* loc2 = mat2->locations[j];
            if (loc1->c == loc2->r) {
                val* val2 = mat2->values[j];
                val *current = coo_get_value(res,
                                             loc1->r,
                                             loc2->c);
                if (current) {
                    *current += *val1 * *val2;
                } else {
                    coo_set_value(res, *val1 * *val2,
                                  loc1->r, loc2->c);
                }
            }
        }
    }

    \end{verbatim}

    \subsubsection{Complexity analysis}

    The addition operation consists of iterating over the values of both
    matrices. If the first matrix has $m_1$ nonzero entries and the second
    matrix has $m_2$ nonzero entries, then there are $m_1 + m_2$ steps. Each
    step consists of value retrieval, single addition and value assignment.
    Because the hash table allows constant time lookup and insertion, then the
    time complexity is $O(m_1+m_2)$.

    The multiplication uses Cartesian product of the hash table entries. Each
    element in the product is iterated over. This requires $m_1 m_2$ steps. In
    each step, two retrievals from the hash table, a single condition
    checking and a single assignment is done. Because the hash table operations
    are constant time, then the time complexity of multiplication is $O(m_1
    m_2)$.

    \subsubsection{Experimental results}

    We experimented on the running time of the algorithms to compare them
    against theoretical running time. The corresponding plots are illustrated in
    Figures~\ref{fig:add} and~\ref{fig:mult}.

    \begin{figure}
        \begin{tikzpicture}
            \begin{axis}
                [xlabel=Number of nonzero points,
                    compat=1.3,
                    ylabel=Running time,
                    height=0.1\textheight,
                    width=\textwidth,
                    xmin=50000,
                    xmax=140000,
                    ymin=0.05,
                    ymax=0.280,
                    grid=major,
                    legend pos=north west,
                    yticklabel style={/pgf/number format/fixed, /pgf/number
                    format/precision=7},
                    xtick={50000, 60000, 70000, 80000, 90000, 100000, 110000,
                    120000, 130000, 140000},
                scaled x ticks=false]
                \addplot+[smooth,color=blue] table[col
                sep=comma,x=algorithm,y=time,color=blue] {add.csv};
                \addlegendentry{Algorithm}
                \addplot+[color=red] table[col
                sep=comma,x=algorithm,y=linear,color=red] {add.csv};
                \addlegendentry{$18/5000000 m$}
            \end{axis}
        \end{tikzpicture}
\label{fig:add}
        \caption{Running time of the addition algorithm}
    \end{figure}

    \begin{figure}
        \begin{tikzpicture}
            \begin{axis}
                [xlabel=Number of nonzero points,
                    compat=1.3,
                    ylabel=Running time,
                    height=0.1\textheight,
                    width=\textwidth,
                    xmin=9000,
                    xmax=101000,
                    ymin=0,
                    ymax=160,
                    grid=major,
                    legend pos=north west,
                    yticklabel style={/pgf/number format/fixed, /pgf/number
                    format/precision=7},
                    xtick={10000, 20000, 30000, 40000, 50000, 60000, 70000,
                    80000, 90000, 100000},
                scaled x ticks=false]
                \addplot+[smooth,color=blue] table[col
                sep=comma,x=algorithm,y=time,color=blue] {mult.csv};
                \addlegendentry{Algorithm}
                \addplot+[smooth,color=red] table[col
                sep=comma,x=algorithm,y=quad,color=red] {mult.csv};
                \addlegendentry{$7/500000000 m^2$}
            \end{axis}
        \end{tikzpicture}
\label{fig:mult}
        \caption{Running time of the multiplication algorithm}
    \end{figure}


    \nocite{bell2009implementing}
    \nocite{bell2008efficient}
    \nocite{dongarraxz1994sparse}
    \nocite{duff1986direct}
    \bibliographystyle{plain}
    \bibliography{../materials/references.bib}

\end{multicols}

\end{document}
